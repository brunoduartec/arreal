\newpage
% \color{white}
% \pagecolor{black}


\ifdefined\useChapters
\chapter{Leve seu relógio}
\else
\chapter{}
\fi
% \begin{quotation}
%     Onde reside a nossa vontade? No nosso corpo? Onde estão nossos desejos? Na cabeça? No coração? Porque o que nos faz sentir medo em uma situação de extremo, não nos desperta nada quando acontece com outra pessoa? Penso, logo existo? Percebo, logo acredito?
% \end{quotation}

% \newpage
% \color{black}
% \pagecolor{white}

Não sei onde tudo isso vai me levar, mas ao que tudo indica, a partir de agora tudo pode acontecer, aliás algo me diz que assim como passei por essas experiências de desdobramento, descobri que é assim que se chama a tal experiência de sair do corpo, e alterar a realidade que experiencio, provavelmente muitas outras maneiras de quebrar o que acreditamos ser o comum, são possíveis, aliás, como posso duvidar de algo e agora que tudo começou, vou até o fim.

Já fazem algumas noites que eu tento desdobrar, mas acho que a ansiedade ou medo tem me afastado de ter sucesso. Confesso que tenho um pouco de receio de que me aconteça de novo, que eu saia do meu corpo e não consiga mais retornar. 

Criar outras realidades me parece então algo mais promissor e menos arriscado, mas mesmo tendo visto o que Crispim realizou, e enquanto estava com ele ter conseguido também, agora tudo parece muito mais difícil. Depois de um tempo cheguei à conclusão de que o que me falta é entender a real natureza do efeito, da última vez foi tudo tão intenso que não tive tempo de compreender.

Já ouvi dizer que podemos aprender as coisas na dor ou no amor, gostaria muito que fosse no amor, mas eu não consigo esperar a vida toda pra descobrir, tudo é muito grande pra ficar esperando, então eu tenho um plano, talvez um tanto suicida, mas eu tenho confiança que se eu estiver em perigo e passando por uma situação de estresse como da última vez, vou conseguir destravar. Amanhã de noite, tudo vai estar resolvido.

Hoje preciso pensar nos detalhes de como vou fazer o que estou pensando e pra isso preciso de um momento de paz, pra conseguir pensar.

Minha irmã sempre fica me pedindo coisas, tem horas que me enche. Depois que eu despertei do meu coma, é verdade que não fui atrás de nada muito concreto na minha vida e ela vive me dizendo que não somos ricos pra ter esses tais anos sabáticos, mas eu discordo, preciso de um tempo pra colocar minhas ideias no lugar e entender tudo o que está acontecendo e o meu lugar no mundo de novo.

- Vai moleque, vai lá no mercadinho buscar algumas coisas pra mim -- Me diz Lúcia, olhando para uma lista na mão.

- Eu? Por que tenho que ir eu?

- E por que não pode? Você é um rei por acaso? Não me lembro de ter visto você ser coroado. Já te disse várias vezes que nessa vida a gente tem que ser protagonista e não ficar aí, sentado esperando, igual você está.

- Você fica sempre com a cabeça nas nuvens, fala ela enquanto confere uma pequena lista.

- Você que é...

- Ah, não tenho tempo pra isso não, vai logo. -- Fala, colocando a lista na minha mão.

- Aposto que você nem tinha notado que perdeu o dia inteiro deitado viajando e o dia já escurecendo.

- Claro que eu tinha notado, na verdade eu estava fazendo varias coisas importantes.

- Então pega essa lista e vai fazer mais uma.

Uma dúzia de ovo, farinha de trigo, um litro de leite, canela e lá vou eu buscar os ingredientes para fazer bolinho de chuva. Realmente, muito importante essa compra.

Nos últimos tempos troquei todos os meus caminhos para passar sempre na frente da casa amarela, tenho andado bem mais para algumas coisas mas pelo menos sempre posso passar por lá e ver se consigo dar uma espiada na Joana. Pra chegar na vendinha do seu Antônio geralmente só precisava andar reto na minha rua por três quadras, mas no novo caminho, viro à primeira direita, desço três ruas, depois viro à esquerda passo em frente à casa amarela após duas ruas e aí só tenho que voltar umas 5 ruas, nem é tanto assim por que no caminho eu vejo muito mais pessoas, conheço o bairro. 

Ao chegar em frente à casa noto que as luzes estavam acessas e podia ver o silueta de Joana, mas logo noto que a sombra parecia um pouco menor e tinha cabelos curtos. Ao lado da sombra mais outras duas um pouco maiores andando pela sala como se estivessem ansiosos. Decido bater na porta por que achei meio estranho aquela movimentação

-- Joana, você tá em casa? -- Grito enquanto bato palmas em frente ao portão de madeira.

Ouço alguns barulhos e logo a porta se abre.

-- Olá, é você? Tudo bem?

-- Oi, então, tá sim -- Falo eu gaguejando um pouco

-- Que bixo te mordeu cara

-- Então, você tinha me falado que era muito solitária e que nunca tinha muito com quem conversar mas parece que eu vi umas sombras pela janela

-- Sombras pela janela? Você tava me espionando? 

-- Não, não tava não, eu estava passando por aqui pra ir na vendinha do seu João e acabei notando

-- A vendinha do seu Antônio que fica a sete quadras daqui e totalmente fora de caminho de onde me disse que morava? Fala a verdade você ta me espionando. -- Fala ela enquanto fecha a porta e vem andando na minha direção

-- Ah, eu agora comecei a fazer caminhadas. -- Falo eu meio constrangido por não ter muito desculpa por estar ali

-- Olha faz o seguinte passa aqui outro dia pra conversar se quiser mas hoje eu já estava indo deitar, estou um pouco cansada. Não que eu te deva nenhuma satisfação mas eu estou totalmente sozinha, você deve estar cansado também e está vendo coisas -- Fala Joana me dando dois tapas no ombro.

Achei muito estranho ela mentir pra mim desse jeito e me falar que não tinha ninguem na sua casa o que aumentou ainda mais a minha suspeita de que tinha alguma coisa mais estranha com toda essa estória do que eu sabia.

Enquanto ando até a vendinha do seu Antônio, reparo nas árvores balançando, nas ruas em que eu cresci e percebo que apesar de estar tudo igual, no mesmo lugar, também estava tudo diferente. Cinco anos são mais que suficientes pra nos desconectar um pouco, é impressionante como perdemos contato com a nossa realidade e o que conhecemos por tão pouco.

O seu Antônio estava exatamente o mesmo senhorzinho de sempre, mas pelo menos agora aparentemente tinha um ajudante, que eu não entendo o que estava fazendo ali, porque enquanto fazia a minha compra, continuava sentado próximo ao caixa, quieto me acompanhando com o olhar.

- Oi rapaz, me disse o ajudante com uma voz calma. 

- Tudo bem com você? Pela sua expressão, vejo que ainda não encontrou tudo o que procura, me disse o atendente ali, de trás do balcão.

- Já encontrei sim, disse pra ele pegando o último pacotinho de canela.

- Não é disso que estou falando, você sempre foi muito impetuoso e não pretendo te convencer de nada, mas se eu puder te dar um conselho, hoje de noite, se lembre de levar seu relógio de bolso.

- Hoje de noite?!!!

Eu não havia falado pra ninguém que faria algo diferente hoje. No susto, minhas sacolas de compra caíram das minhas mãos, abaixei para pegá-las e quando retornei, ele já não estava mais lá, tinha desaparecido.

- Seu Antônio, onde está o rapaz que estava aqui? 

- Que rapaz? Não tinha ninguém aqui, me disse enquanto continuava a contar as moedinhas do caixa.

Bom, ele não era um ajudante então e das duas uma, ou ele não percebeu o rapaz por estar tão entretido com as moedas, afinal seu Antônio já é um senhorzinho, ou realmente não tinha outra pessoa ali além de nós dois.

Não perguntei duas vezes, porque minha imagem na cidade já não era das melhores, muitas pessoas tinham de mim essa impressão de alguém que sempre faz perguntas estranhas e está atrás de coisas que ninguém acha relevante. 

Enquanto volto pra casa, vou olhando pelas ruas pra ver se encontro aquele rapaz de novo. Como minha irmã disse, preciso ser protagonista da minha história.

Mais uma experiência inesperada que não compreendo, mas dados os últimos acontecimentos, é bom acreditar no que o rapaz me disse, levar meu relógio e um tanto de receio, afinal, não é todo dia que alguém aparece, fala umas frases e desaparece. 

Depois disso, preciso pensar com um pouco mais de cuidado em como vou colocar meu plano em prática porque hoje, a noite promete.


